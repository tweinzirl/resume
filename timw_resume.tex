\documentclass{article}
\usepackage[cm]{fullpage}
\usepackage{color}
\usepackage{hyperref}

\hypersetup{breaklinks=true,%
pagecolor=white,%
colorlinks=true,%
linkcolor=cyan,%
urlcolor=MyDarkBlue}

\definecolor{MyDarkBlue}{rgb}{0,0.0,0.45}

%%%%%%%%%%%%%%%%%%%%%%%%%%
% Formatting parameters  %
%%%%%%%%%%%%%%%%%%%%%%%%%%

\newlength{\tabin}
\setlength{\tabin}{1em}
\newlength{\secsep}
\setlength{\secsep}{0.1cm}

\setlength{\parindent}{0in}
\setlength{\parskip}{0in}
\setlength{\itemsep}{0in}
\setlength{\topsep}{0in}
\setlength{\tabcolsep}{0in}

\definecolor{contactgray}{gray}{0.3}
\pagestyle{empty}

%%%%%%%%%%%%%%%%%%%%%%%%%%
%  Template Definitions  %
%%%%%%%%%%%%%%%%%%%%%%%%%%

\newcommand{\lineunder}{\vspace*{-8pt} \\ \hspace*{-6pt} \hrulefill \\ \vspace*{-15pt}}
\newcommand{\name}[1]{\begin{center}\textsc{\Huge#1}\\\end{center}}
\newcommand{\program}[1]{\begin{center}\textsc{#1}\end{center}}
\newcommand{\contact}[1]{\begin{center}\color{contactgray}{\small#1}\end{center}}

\newenvironment{tabbedsection}[1]{
  \begin{list}{}{
      \setlength{\itemsep}{0pt}
      \setlength{\labelsep}{0pt}
      \setlength{\labelwidth}{0pt}
      \setlength{\leftmargin}{\tabin}
      \setlength{\rightmargin}{\tabin}
      \setlength{\listparindent}{0pt}
      \setlength{\parsep}{0pt}
      \setlength{\parskip}{0pt}
      \setlength{\partopsep}{0pt}
      \setlength{\topsep}{#1}
    }
  \item[]
}{\end{list}}

\newenvironment{nospacetabbing}{
    \begin{tabbing}
}{\end{tabbing}\vspace{-1.2em}}

\newenvironment{resume_header}{}{\vspace{0pt}}


\newenvironment{resume_section}[1]{
  \filbreak
  \vspace{2\secsep}
  \textsc{\large#1}
  \lineunder
  \begin{tabbedsection}{\secsep}
}{\end{tabbedsection}}

\newenvironment{resume_subsection}[2][]{
  \textbf{#2} \hfill {\footnotesize #1} \hspace{2em}
  \begin{tabbedsection}{0.5\secsep}
}{\end{tabbedsection}}

\newenvironment{subitems}{
  \renewcommand{\labelitemi}{-}
  \begin{itemize}
      \setlength{\labelsep}{1em}
}{\end{itemize}}

\newenvironment{resume_employer}[4]{
  \vspace{\secsep}
  \textbf{#1} \\ 
  \indent {\small #2} \hfill {\footnotesize#3 (#4)}
  \begin{tabbedsection}{0pt}
  \begin{subitems}
}{\end{subitems}\end{tabbedsection}}


%%%%%%%%%%%%%%%%%%%%%%%%%%
%     Start Document     %
%%%%%%%%%%%%%%%%%%%%%%%%%%

\begin{document}

\begin{resume_header}
\name{Tim Weinzirl, Ph.D.}
%\program{Astrophysicist, Ph.D.}
\contact{Email: \href{mailto: tweinzirl@gmail.com}{tweinzirl@gmail.com} \hspace{0.1cm} Phone: 510-725-9054 \hspace{0.1cm} Online materials: \href{http://loveofdatascience.blogspot.com/}{Personal projects}, \href{https://github.com/tweinzirl}{Github} \hspace{0.1cm} Status: U.S. citizen}
\end{resume_header}

\begin{resume_section}{Skills}
  \begin{nospacetabbing}
  \textbf{Programming:}  \= Python (2006--Present), SQL (2017-Present), Scala (2023--Present), R (2015--2017), C (2003--2007) \\* %, Java (2003--2006)
  \textbf{Select tools:} \> Dash, Databricks, Docker, Git, LangChain, Linux, OpenAI, OpenShift, Snowflake, TensorFlow \\*
  \end{nospacetabbing}

\end{resume_section}

\begin{resume_section}{Work Experience}

  \begin{resume_employer}{Data Scientist / Senior Data Scientist}{First Republic Bank (\textit{now part of JPMorgan Chase})}
  {San Francisco, California}{June 2017 -- Present}

    \item \texttt{AI}: Prototyped an LLM-agent for use by the sales team and analysts; pitched to executives and business lines. Developed machine learning models (e.g., grading loan applications, client segmentation, fuzzy matching) according to model governance standards and deployed them as self-service web (e.g., Dash) applications.
    \item \texttt{Product operations:} Authored/maintained SQL+Python data pipelines for producing and delivering daily content (opportunities, reports) to salespersons and executives over web and email channels. Prepared bootstrap metrics (e.g., deposit growth) to demonstrate product efficacy to executives.
    \item \texttt{Quantitative analysis:} Made ad hoc data-driven recommendations to management, often on short notice. Produced targeted client lists for use in sales campaigns that led to millions of dollars in new deposits (e.g., the Spring 2023 CD reinstatement campaign raised \underline{\textbf{\$14M}}).
    \item \texttt{Technical leadership:} Contributed uniquely to the infrastructure of the broader enterprise (e.g., tailored Docker images; custom Python packages for data analysis, database/file I/O, fuzzy matching). Prepared/presented learning materials (e.g., a \underline{\textbf{12-week}} Python course) to spread technical excellence. Routinely carried out R\&D to solve challenges (e.g., I/O with cloud databases, sending emails with Python) concerning new initiatives. \underline{\textbf{When someone had a Python question, they generally came to me first.}}
% (e.g., ``How do I install TensorFlow?'', ``How do I create a Python package?'')
  \end{resume_employer}

  \begin{resume_employer}{Research Fellow}{University of Nottingham}
  {Nottingham, England}{September 2014 -- March 2017}
    \item Led research tasks (e.g., parameter estimation with Bayesian Markov Chain Monte Carlo techniques for \underline{\textbf{$\sim$22,000}} spectra, hypothesis testing with Kolmogorov-Smirnov and chi-squared statistical tests, regression, and coding data pipelines) that facilitated \underline{\textbf{nine}} (two first-author, one second-author) journal publications.
    %\item Engineered a new \href{https://arxiv.org/pdf/1508.06831.pdf}{sky subtraction algorithm} for Fabry-P\'erot images involving median filtering in polar coordinates. This approach preserved $\sim37\%$ more light in galaxy outskirts than the initial naive method.
    %\item Trained an undergraduate summer research student who was subsequently hired for a Ph.D. position.
  \end{resume_employer}

  \begin{resume_employer}{Data Science Advisor, Software Engineer}{\href{http://peopleanalyst.com}{People Analyst}}
  %{Austin, TX and Remote}{January 2014 -- December 2015}
  {Austin, TX and Remote}{January 2014 -- September 2016}
   \item \texttt{Linkedin.com web scraping:} Wrote/deployed a Linkedin web crawler with Scrapy+Selenium WebDriver. Retrieved \underline{\textbf{1,489 profiles}} for persons working in the HR departments of children's hospitals across the US.
   %\item \texttt{Tim's Baby Pythons:} Built a pipeline to put structured text from Google Spreadsheets into Tumblr blog posts using APIs from Google and Tumblr.
    %  Automatically created 1635 posts, saving $\sim50$ human hours.
   \item \texttt{Company roster simulation:} Developed an R Shiny \href{https://peopleflow.shinyapps.io/Roster}{web application} to simulate monthly staff rosters given organizational characteristics (e.g., size, attrition rate, gender gap).
 %\item \texttt{Job posting analysis:} Mined job posting text from Github and Stack 
 %  Overflow to track which specific technologies (e.g., programming languages, database systems) were in demand by companies. %Deployed a Docker process to capture job postings from the Stack Overflow RSS feed.
  %\item \texttt{Simply Hired web scraping:} Built a Docker process to query simplyhired.com via HTTP, extract job listings from an XML feed, and then archive the postings in a Google BigQuery table.  Deployed on Iron.io.
   %\item \texttt{Data analysis:} Analyzed employee questionnaire data using scatter plots, heat maps, and the Pearson and Kendall correlation coefficients 
   %to identify factors 
%(e.g., leadership, compensation, job family, office location) 
%correlating with employee ``engagement''.
  \end{resume_employer}

  \begin{resume_employer}{Graduate Research Fellow/Postdoctoral Researcher}{University of Texas at Austin}
  {Austin, TX}{August 2006 -- August 2014}
    %\item  Wrote 9 papers (3 as 1$^{st}$ author) as part of four international teams to earn an M.A. and Ph.D. in Astronomy. 
    \item Conducted scientific computing/analysis tasks (e.g., regression with parametric models, constructing mock galaxy images, interpreting real vs simulated data) to publish nine (\underline{\textbf{three first-author}}) papers, earn Ph.D. 
    %\item Collaborated with a distributed team to build a pipeline for synthesizing datacubes from 10s of GB of data. % from spatially resolved spectral data.
    \item As co-principle investigator of the international 
\href{https://users.obs.carnegiescience.edu/gblancm/venga.html}{VENGA}
project, led five written proposals responsible for earning 101 nights (\underline{\textbf{$\sim$60\%}} of VENGA's total allocated time) on
    the 2.7m telescope at McDonald Observatory.
    %\item Trained students in scientific computing and in using irreplaceable equipment at McDonald Observatory.
  \end{resume_employer}



\end{resume_section}

\iffalse %remove this section for now
\begin{resume_section}{Personal Data Science Projects}
\begin{subitems}

\item Created a \href{http://loveofdatascience.blogspot.com/}{blog} to highlight self-guided data science projects
involving machine learning (e.g., 
\href{http://loveofdatascience.blogspot.co.uk/2014/01/gender-classification-with-machine.html}{gender classification}, 
\href{http://loveofdatascience.blogspot.co.uk/2016/07/naive-bayes-text-classification-of.html}{document classification}), 
text mining (e.g., \href{http://loveofdatascience.blogspot.co.uk/2016/05/how-to-save-earth-with-word-association.html}{word associations and topic modeling}), 
\href{http://loveofdatascience.blogspot.co.uk/2014/09/web-crawling-for-job-postings.html}{web scraping},
\href{http://loveofdatascience.blogspot.co.uk/2015/12/does-congressional-approval-rating-vary.html}{hypothesis testing}, and data visualization (e.g., 
\href{http://loveofdatascience.blogspot.co.uk/2016/09/graph-analysis-of-leaked-democratic.html}{network maps with an R Shiny web application}). 

\end{subitems}
\end{resume_section}
\fi

\begin{resume_section}{Education}

  \begin{resume_subsection}[San Francisco, CA (Spring, 2017)]{The Data Incubator}
      \begin{subitems}
          %\item During this eight-week data science fellowship, I received training in essential data science technologies (SQL, Python, parallel computing) and built \href{http://tranquil-ravine-75736.herokuapp.com/about.html}{a recommender system} for learning new technical skills.
          \item During this eight-week data science \underline{\textbf{fellowship}}, I received training in essential data science technologies (SQL, Python, distributed computing) and built a recommender system for learning new technical skills.
      \end{subitems}
  \end{resume_subsection}

  \begin{resume_subsection}[Austin, TX (2006--2013)]{University of Texas at Austin}
      \begin{subitems}
          \item Ph.D. in Astronomy, 2013; M.A. in Astronomy, 2008
      \end{subitems}
  \end{resume_subsection}

  %\begin{resume_subsection}[Des Moines, IA (2002--2006)]{Drake University}
  %    \begin{subitems}
  %        \item Bachelor of Science, Physics, Astronomy, Math (minor)
  %    \end{subitems}
  %\end{resume_subsection}
\end{resume_section}

\begin{resume_section}{Awards \& Honors}
\begin{subitems}

\item Ph.D. thesis was 
\href{http://www.springer.com/us/book/9783319069586}{published (ISBN 978-3-319-06959-3)}
in 2014 as its own volume in Springer's \href{http://www.springer.com/series/8790}{book series} for recognizing \underline{\textbf{outstanding Ph.D. research}}. 
%\item Rodger Doxsey Travel Prize (\$400) to enable the oral presentation of dissertation research at the American Astronomical Society's 221st meeting, 2012 (given to the top 10\% of dissertation speakers).
\item Outstanding Master's Thesis Award (\underline{\textbf{\$1,000}}) in the College of Natural Science and College of Engineering at the University of Texas at Austin, 2009 (one awarded per year).

\end{subitems}
\end{resume_section}

\end{document}
